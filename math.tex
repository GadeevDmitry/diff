\documentclass[12pt,a5paper,fleqn]{article}
\usepackage[utf8]{inputenc}
\usepackage{amssymb, amsmath, multicol}
\usepackage[russian]{babel}
\usepackage{graphicx}
\usepackage[shortcuts,cyremdash]{extdash}
\usepackage{wrapfig}
\usepackage{floatflt}
\usepackage{lipsum}
\usepackage{concmath}
\usepackage{euler}
\usepackage{libertine}

\oddsidemargin=-15.4mm
\textwidth=127mm
\headheight=-32.4mm
\textheight=277mm
\tolerance=100
\parindent=0pt
\parskip=8pt
\pagestyle{empty}

\usepackage[normalem]{ulem}
\usepackage{mdframed}
\usepackage{amsthm}

\flushbottom

\begin{document}
\begin{center}
{\bf \Large
Введение в дифференцирование.
}
\end{center}
Здравствуйте, дорогие школьники, рады вас приветствовать на этом замечательном курсе, который даст Вам быстрый старт в спортивное дифференцирование. Этот курс подойдёт всем, в независимости от начальных знаний и умений. Разбирая примеры разной сложности шаг за шагом, мы уверены, что даже самый неспособный ученик сможет разобраться с материалом.


\newpage
{\bf \Large
Задача 1
}

$$
f(x)=5* x+8
$$
В данном примере перед нами многочлен первой степени. Воспользовавшись правилом взятия производной степенной функции, получаем

$$
f'(x)=(5* x+8)'=0* x+5*1+0=5
$$
Заметим, что производная константы равна нулю, поэтому если бы вместо цифры 8 в примере №1 стояла какая-нибудь другая цифра(например 1, 2, 3, или даже 4), то производная не изменилась бы.

{\bf \Large
Задача 2
}

$$
f(x)=\left(5^{ ln\left(8\right)}+ sh\left( sin\left(\frac{344}{ arccos\left(-0.78\right)}\right)\right)\right)* x+e
$$
Раз уж мы уже научились брать производные многочленов, пример №2 точно не составит труда - здесь всё то же самое, за исключением того что нужно немного заняться счётом. Я думаю, на данном этапе нашего курса это может сделать даже 1.5-летний ребёнок. Упрощая, получается:

$$
f(x)=\left(5^{ ln\left(8\right)}+ sh\left( sin\left(\frac{344}{ arccos\left(-0.78\right)}\right)\right)\right)* x+e=29.5285* x+e
$$
Осталось только взять производную:

$$
f'(x)=(29.5285* x+e)'=0* x+29.5285*1+0=29.5285
$$

\newpage
{\bf \Large
Задача 3
}

$$
f(x)=\left(2* x- x^{6}\right)*\sqrt{1-3* x^{5}}
$$
Давайте немного отвлечёмся от скучного дифференцирования и устного счёта. Для этого предлагаю разложить незамысловатую функцию в ряд Тейлора в окрестности -2.
Как известно(смотри пример №1), любую функцию f(x) в точке $x_{0}$ можно округлить до $o((x-x_{0})^{n})$ по формуле:

$$
f(x_{0})=\sum_{i=0}^{n}\frac{f^{(i)}(x_{0})x^i}{i!} + o((x-x_{0})^{n})
$$
Зафиксируем n = 3. Тогда:

$$
f(x)=\left(2* x- x^{6}\right)*\sqrt{1-3* x^{5}}
$$

$$
f(-2)=\left(2*(-2)-(-2)^{6}\right)*\sqrt{1-3*(-2)^{5}}=-669.722
$$

$$
f'(x)= \alpha+ \gamma
$$
, где

$$
\alpha = \left(2- x^{5}*6\right)*\sqrt{1-3* x^{5}}
$$

$$
\beta = 2*\sqrt{1-3* x^{5}}
$$

$$
\gamma = \left(2* x- x^{6}\right)*\left(\frac{-3*\left( x^{4}*5\right)}{ \beta}\right)
$$

$$
f'(-2)=1910.68+828.522=2739.2
$$

\newpage

$$
f''(x)= \alpha+ \gamma+\left( \lambda+\left(2* x- x^{6}\right)*\left(\frac{ \mu*\left(2*\sqrt{1-3* x^{5}}\right)-\left(-3*\left( x^{4}*5\right)\right)*\left(2*\left(\frac{-3*\left( x^{4}*5\right)}{2*\sqrt{1-3* x^{5}}}\right)\right)}{\left(2*\sqrt{1-3* x^{5}}\right)^{2}}\right)\right)
$$
, где

$$
\alpha = \left(- x^{4}*5*6\right)*\sqrt{1-3* x^{5}}
$$

$$
\beta = 2*\sqrt{1-3* x^{5}}
$$

$$
\gamma = \left(2- x^{5}*6\right)*\left(\frac{-3*\left( x^{4}*5\right)}{ \beta}\right)
$$

$$
\lambda = \left(2- x^{5}*6\right)*\left(\frac{-3*\left( x^{4}*5\right)}{ \beta}\right)
$$

$$
\mu = -3*\left( x^{3}*4*5\right)
$$

$$
f''(-2)=(-4727.45)+(-2363.73)+\left((-2363.73)+\left(2*(-2)-(-2)^{6}\right)*\left(\frac{480*\left(2*\sqrt{1-3*(-2)^{5}}\right)-\left(-3*\left((-2)^{4}*5\right)\right)*\left(2*\left(\frac{-3*\left((-2)^{4}*5\right)}{2*\sqrt{1-3*(-2)^{5}}}\right)\right)}{\left(2*\sqrt{1-3*(-2)^{5}}\right)^{2}}\right)\right)=-10087
$$
\end{document}
