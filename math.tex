\documentclass[12pt,a4paper,fleqn]{article}
\usepackage[utf8]{inputenc}
\usepackage{amssymb, amsmath, multicol}
\usepackage[russian]{babel}
\usepackage{graphicx}
\usepackage[shortcuts,cyremdash]{extdash}
\usepackage{wrapfig}
\usepackage{floatflt}
\usepackage{lipsum}
\usepackage{concmath}
\usepackage{euler}
\usepackage{libertine}

\oddsidemargin=-15.4mm
\textwidth=180mm
\headheight=-32.4mm
\textheight=277mm
\tolerance=100
\parindent=0pt
\parskip=8pt
\pagestyle{empty}

\usepackage[normalem]{ulem}
\usepackage{mdframed}
\usepackage{amsthm}

\flushbottom

\begin{document}
\begin{center}
{\bf \Large
Введение в дифференцирование.
}
\end{center}
Здравствуйте, дорогие школьники, рады вас приветствовать на этом замечательном курсе, который даст Вам быстрый старт в дифференцирование. Этот курс подойдёт всем, в независимости от начальных знаний и умений. Разбирая примеры разной сложности шаг за шагом, мы уверены, что даже самый неспособный ученик сможет разобраться с материалом.


\newpage
{\bf \Large
Задача 1
}

$$
f(x)=5\cdot x+8
$$
В данном примере перед нами многочлен первой степени. Воспользовавшись правилом взятия производной степенной функции, получаем

$$
f'(x)=(5\cdot x+8)'=0\cdot x+5\cdot1+0=5
$$
Заметим, что производная константы равна нулю, поэтому если бы вместо цифры 8 в примере №1 стояла какая-нибудь другая цифра(например 1, 2, 3, или даже 4), то производная не изменилась бы.

{\bf \Large
Задача 2
}

$$
f(x)=\left(5^{ ln8}+ sh\left( sin\left(\frac{344}{ arccos\left(-0.78\right)}\right)\right)\right)\cdot x+e
$$
Раз уж мы уже научились брать производные многочленов, пример №2 точно не составит труда - здесь всё то же самое, за исключением того что нужно немного заняться счётом. Я думаю, на данном этапе нашего курса это может сделать даже 1.5-летний ребёнок. Упрощая, получается:

$$
f(x)=\left(5^{ ln8}+ sh\left( sin\left(\frac{344}{ arccos\left(-0.78\right)}\right)\right)\right)\cdot x+e=29.5285\cdot x+e
$$
Осталось только взять производную:

$$
f'(x)=(29.5285\cdot x+e)'=0\cdot x+29.5285\cdot1+0=29.5285
$$
{\bf \Large
Задача 3
}

$$
f(x)=\left(2\cdot x- x^{6}\right)\cdot\sqrt{1-3\cdot x^{5}}
$$
Давайте немного отвлечёмся от скучного дифференцирования и устного счёта. Для этого предлагаю разложить незамысловатую функцию в ряд Тейлора в окрестности -3.
Как известно(смотри пример №1), любую функцию f(x) в окрестности точки $x_{0}$ можно округлить до $o((x-x_{0})^{n})$ по формуле:

$$
f(x_{0})=\sum_{i=0}^{n}\frac{f^{(i)}(x_{0})x^i}{i!} + o((x-x_{0})^{n})
$$
Зафиксируем n = 2. Тогда:

$$
f(x)=\left(2\cdot x- x^{6}\right)\cdot\sqrt{1-3\cdot x^{5}}
$$

$$
f(-2)=\left(2\cdot(-2)-(-2)^{6}\right)\cdot\sqrt{1-3\cdot(-2)^{5}}=-669.722
$$
Так как $\pi^{e}<e^{\pi}$, то:
$$
f'(x)= \alpha+ \gamma
$$
, где

$$
\alpha = \left(2- x^{5}\cdot6\right)\cdot\sqrt{1-3\cdot x^{5}}
$$

$$
\beta = 2\cdot\sqrt{1-3\cdot x^{5}}
$$

$$
\gamma = \left(2\cdot x- x^{6}\right)\cdot\left(\frac{-3\cdot\left( x^{4}\cdot5\right)}{ \beta}\right)
$$

$$
f'(-2)=1910.68+828.522=2739.2
$$
И ежу очевидно, что:
$$
f''(x)= \alpha+ \gamma+\left( \gamma+ \theta\right)
$$
, где

$$
\alpha = \left(- x^{4}\cdot5\cdot6\right)\cdot\sqrt{1-3\cdot x^{5}}
$$

$$
\beta = 2\cdot\sqrt{1-3\cdot x^{5}}
$$

$$
\gamma = \left(2- x^{5}\cdot6\right)\cdot\left(\frac{-3\cdot\left( x^{4}\cdot5\right)}{ \beta}\right)
$$

$$
\delta = -3\cdot\left( x^{3}\cdot4\cdot5\right)
$$

$$
\varepsilon = 2\cdot\left(\frac{-3\cdot\left( x^{4}\cdot5\right)}{ \beta}\right)
$$

$$
\eta = \left(-3\cdot\left( x^{4}\cdot5\right)\right)\cdot \varepsilon
$$

$$
\theta = \left(2\cdot x- x^{6}\right)\cdot\left(\frac{ \delta\cdot \beta- \eta}{ \beta^{2}}\right)
$$

$$
f''(-2)=(-4727.45)+(-2363.73)+\left((-2363.73)+(-632.069)\right)=-10087
$$
В итоге, получаем:
$$
f(x)=(-669.722)+2739.2x+(-10087)\frac{x^2}{2}+o((x+3)^{2})
$$
\end{document}
