\documentclass[12pt,a4paper,fleqn]{article}
\usepackage[utf8]{inputenc}
\usepackage{amssymb, amsmath, multicol}
\usepackage[russian]{babel}
\usepackage{graphicx}
\usepackage[shortcuts,cyremdash]{extdash}
\usepackage{wrapfig}
\usepackage{floatflt}
\usepackage{lipsum}
\usepackage{concmath}
\usepackage{euler}
\usepackage{libertine}

\oddsidemargin=-15.4mm
\textwidth=180mm
\headheight=-32.4mm
\textheight=277mm
\tolerance=100
\parindent=0pt
\parskip=8pt
\pagestyle{empty}

\usepackage[normalem]{ulem}
\usepackage{mdframed}
\usepackage{amsthm}

\flushbottom

\begin{document}
\begin{center}
{\bf \Large
Введение в дифференцирование.
}
\end{center}
Здравствуйте, дорогие школьники, рады вас приветствовать на этом замечательном курсе, который даст Вам быстрый старт в дифференцирование. Этот курс подойдёт всем, в независимости от начальных знаний и умений. Разбирая примеры разной сложности шаг за шагом, мы уверены, что даже самый неспособный ученик сможет разобраться с материалом.


\newpage
{\bf \Large
Задача 1
}
В данном примере перед нами многочлен первой степени. Воспользовавшись правилом взятия производной степенной функции, получаем

$$
f(x)=5\cdot x+8
$$

$$
f'(x)=(5\cdot x+8)'=0\cdot x+5\cdot1+0=5
$$
Заметим, что производная константы равна нулю, поэтому если бы вместо цифры 8 в примере №1 стояла какая-нибудь другая цифра(например 1, 2, 3, или даже 4), то производная не изменилась бы.

{\bf \Large
Задача 2
}
Раз уж мы уже научились брать производные многочленов, пример №2 точно не составит труда - здесь всё то же самое, за исключением того что нужно немного заняться счётом. Я думаю, на данном этапе нашего курса это может сделать даже 1.5-летний ребёнок.

$$
f(x)=\left(5^{ ln8}+ sh\left( sin\left(\frac{344}{ arccos\left(-0.78\right)}\right)\right)\right)\cdot x+e
$$
Упростив, получаем.

$$
f(x)=\left(5^{ ln8}+ sh\left( sin\left(\frac{344}{ arccos\left(-0.78\right)}\right)\right)\right)\cdot x+e=29.5285\cdot x+e
$$
Производная:
$$
f'(x)=(29.5285\cdot x+e)'=0\cdot x+29.5285\cdot1+0=29.5285
$$
{\bf \Large
Задача 3
}
Давайте немного отвлечёмся от скучного дифференцирования и устного счёта. Для этого предлагаю разложить незамысловатую функцию в ряд Тейлора в окрестности -3.
Как известно(смотри пример №1), любую функцию f(x) в окрестности точки $x_{0}$ можно округлить до $o((x-x_{0})^{n})$ по формуле:

$$
f(x_{0})=\sum_{i=0}^{n}\frac{f^{(i)}(x_{0})x^i}{i!} + o((x-x_{0})^{n})
$$

$$
f(x)=\left(2\cdot x- x^{6}\right)\cdot\sqrt{1-3\cdot x^{5}}
$$

$$
f^{(0)}(x)=\left(2\cdot x- x^{6}\right)\cdot\sqrt{1-3\cdot x^{5}}
$$

$$
f^{(0)}(-3)=\left(2\cdot(-3)-(-3)^{6}\right)\cdot\sqrt{1-3\cdot(-3)^{5}}=-19858.6
$$

$$
f^{(1)}(x)=\left(2- x^{5}\cdot6\right)\cdot\sqrt{1-3\cdot x^{5}}+\left(2\cdot x- x^{6}\right)\cdot\left(\frac{-3\cdot\left( x^{4}\cdot5\right)}{2\cdot\sqrt{1-3\cdot x^{5}}}\right)
$$

$$
f^{(1)}(-3)=\left(2-(-3)^{5}\cdot6\right)\cdot\sqrt{1-3\cdot(-3)^{5}}+\left(2\cdot(-3)-(-3)^{6}\right)\cdot\left(\frac{-3\cdot\left((-3)^{4}\cdot5\right)}{2\cdot\sqrt{1-3\cdot(-3)^{5}}}\right)=55973.2
$$

$$
f^{(2)}(x)= x_{0}+ x_{2},
$$
где
$$
x_{0} = \left(- x^{4}\cdot5\cdot6\right)\cdot\sqrt{1-3\cdot x^{5}}+\left(2- x^{5}\cdot6\right)\cdot\left(\frac{-3\cdot\left( x^{4}\cdot5\right)}{2\cdot\sqrt{1-3\cdot x^{5}}}\right)
$$

$$
x_{1} = \left(-3\cdot\left( x^{3}\cdot4\cdot5\right)\right)\cdot\left(2\cdot\sqrt{1-3\cdot x^{5}}\right)-\left(-3\cdot\left( x^{4}\cdot5\right)\right)\cdot\left(2\cdot\left(\frac{-3\cdot\left( x^{4}\cdot5\right)}{2\cdot\sqrt{1-3\cdot x^{5}}}\right)\right)
$$

$$
x_{2} = \left(2- x^{5}\cdot6\right)\cdot\left(\frac{-3\cdot\left( x^{4}\cdot5\right)}{2\cdot\sqrt{1-3\cdot x^{5}}}\right)+\left(2\cdot x- x^{6}\right)\cdot\left(\frac{ x_{1}}{\left(2\cdot\sqrt{1-3\cdot x^{5}}\right)^{2}}\right)
$$

$$
f^{(2)}(-3)=(-98482.5)+(-41109.4)=-139592
$$

$$
f(x)=(-19858.6)\frac{x^0}{0!}+55973.2\frac{x^1}{1!}+(-139592)\frac{x^2}{2!}
$$
{\bf \Large
Задача 4
}
Геометрический смысл производной.
$$
f(x)=\frac{ ch x\cdot sh x}{ arctan\left(\sqrt{ x-1}\right)\cdot x^{ x}}
$$
Функция:

$$
f(x)=\frac{ ch x\cdot sh x}{ arctan\left(\sqrt{ x-1}\right)\cdot x^{ x}}
$$
Производная:

$$
f'(x)=\frac{ x_{0}- x_{2}}{\left( arctan\left(\sqrt{ x-1}\right)\cdot x^{ x}\right)^{2}},
$$
где
$$
x_{0} = \left( sh x\cdot sh x+ ch x\cdot ch x\right)\cdot\left( arctan\left(\sqrt{ x-1}\right)\cdot x^{ x}\right)
$$

$$
x_{1} = \frac{\frac{1}{2\cdot\sqrt{ x-1}}}{1+\left(\sqrt{ x-1}\right)^{2}}
$$

$$
x_{2} =  ch x\cdot sh x\cdot\left( x_{1}\cdot x^{ x}+ arctan\left(\sqrt{ x-1}\right)\cdot\left( x^{ x}\cdot\left(\frac{ x}{ x}+ ln x\right)\right)\right)
$$
Касательная в точке 3:
$$
g(x)=3.91015+(-0.867862)\cdot\left( x-3\right)
$$
\begin{figure}[h]
\centering
\includegraphics[width=0.5\linewidth]{math_png/image0.png}
\end{figure}\end{document}
